% Copyright (C) 2007 Dominik Dahlem <Dominik.Dahlem@gmail.com>
%
% This file is free software; as a special exception the author gives
% unlimited permission to copy and/or distribute it, with or without
% modifications, as long as this notice is preserved.
%
% This program is distributed in the hope that it will be useful, but
% WITHOUT ANY WARRANTY, to the extent permitted by law; without even
% the implied warranty of MERCHANTABILITY or FITNESS FOR A PARTICULAR
% PURPOSE.

\documentclass{article}
\usepackage[T1]{fontenc}
\usepackage[latin1]{inputenc}

\usepackage{listings}
\def\ccode#1{
  \lstinline[basicstyle=\ttfamily,language=C]{#1} }

\usepackage{fancyheadings}
\pagestyle{fancy}


\author{Dominik Dahlem}
\title{Assignment 1: Linear Solvers}

\begin{document}
\maketitle

\section{General Approach}
\label{sec:general-approach}
\begin{itemize}
\item GNU autotools is used to maintain this project:\\
  autoconf version 2.61 automake version 1.10
\item For all but the last task getopt is used to parse command-line
  parameters.\\
  A help message can be displayed by specifying "-?" or "-h" to the
  command-line of the executable.
\item The GSL library was chosen to provide the API for the vector and
  matrix operations. These entail simple operators within the
  \ccode{gsl_vector} and \ccode{gsl_matrix} packages, but also more
  complex operators in the gsl-cblas library. Consequently, the
  \ccode{gsl_vector} and \ccode{gsl_matrix} containers are heavily
  used to specify vectors and matrices in the assignment respectively.
\item The cunit library was used to implement the unit tests.
\end{itemize}

\section{Configuration}
\label{sec:configuration}
\begin{itemize}
\item The assignment ships with a configure script generated by
  autotools. The following options are supported:
  \begin{itemize}
  \item --enable-test : enables testing using cunit. The configure
    script will check whether the cunit library is present and set a
    automake variables appropriately.
  \item --enable-gcov : enables the coverage analysis using gcov and
    lcov. The configure script will check whether those libraries is
    present and set a automake variables appropriately.
  \end{itemize}
\end{itemize}

\section{Make}
\label{sec:make}
\begin{itemize}
\item Installing the application was not considered.
\item To build the project with MPI support do:
  \begin{enumerate}
  \item No gcov, no tests\\
    ./configure \\
    make \\
    ./src/c/main/linsolvmain -f test.dat -s q -e \\
    Execute the main using the matrix and vector specifications in the
    test.dat file. The solver being invoked is the QR decomposition
    and the eigenvalues are calculated as well.
  \item No gcov\\
    ./configure --enable-test\\
    make check\\
    This command will compile everything and run the cunit tests.
  \item Enable gcov and testing\\
    ./configure --enable-test --enable-gcov\\
    make lcov This command will compile everything, run the cunit
    tests, and generate a snazzy HTML coverage report using lcov
    \footnote{http://ltp.sourceforge.net/coverage/lcov.php}.
  \end{enumerate}
\end{itemize}

\section{Assignment Layout}
\label{sec:assignment-layout}

The assignment contains the octave-code in src/octave and the c-code
in src/c. The octave library is packaged into an octave library called
linearsolv using the octave-forge conventions for packages. The
documentation for the octave code is done with texinfo within the
sources following the conventions of octave-forge. However, for that
reason the texinfo documentation is not generated, because this
package is not part of the octave-forge distribution and I did not
copy the build environment over.

\begin{verbatim}
   - src
      - c
      - octave
   - report
\end{verbatim}

A Doxygen configuration file is provided to generate the code
documentation in HTML. doxygen support is integrated into the
makefiles.  Run: make doxygen-doc

\begin{verbatim}
   - doc
      - doxygen
         - html
\end{verbatim}

The generated doxygen report details the inter-relationships between
the implemented modules and the source files.

The lcov coverage report is provided in the coverage folder.


\section{Remarks on the Approach}
\label{sec:remarks-approach}

The application accepts a number of command-line arguments (see help
output) to configure the file with the matrix and vector
specifications; the linear solver to be used; whether eigenvalues
should be generated; whether the matrix should be generated with a
given dimension; and whether the generated matrix with its solution
should be written into a file called output.dat.

The command-line arguments are parsed with getopt and verified once
they are parsed. The verification makes sure that either a filename or
the dimension of a matrix is specified. Further checks include whether
the dimension is negative, and that the specified solver exists. Any
mistake on the command-line will be reported and the help will be
displayed.

The c-sources for the linear solvers commented out the check whether
the given matrix A is positive definite, because I found an
inconsistency between the GSL library and octave. The
\ccode{matrix_type} method of octave returned that the matrix in
test.dat in the root directory is SPD. However, the GSL library would
report an error. I followed Gilbert Strang's checks in his
``Introduction to Linear Algebra'' book that all upper left
determinants are positive.

The eigenvalue solver uses the direct QR decomposition method without
any pre-processing step such as tridiagonalisation of the matrix A
using the Householder algorithm. Since the QR decomposition can be
used as well to solve the linear system, this is an additional method
provided which can be specified on the command-line.

Since the matrices and vectors are in-memory representations using the
\ccode{gsl_vector} and \ccode{gsl_matrix} containers the respective
sizes are limited by the constraints of the available computer main
memory.

I extended the I/O module a little to read/write the matrices A, x,
and b. The c module is compatible with the octave text output.

\end{document}
%%% Local Variables:
%%% mode: latex
%%% TeX-master: t
%%% End:
